\documentclass{scrreprt}
\usepackage{listings}
\usepackage{underscore}
\usepackage[bookmarks=true]{hyperref}
\usepackage[utf8]{inputenc}
\usepackage[english]{babel}

\usepackage{tcolorbox}
\usepackage{hyperref}

\hypersetup{
    bookmarks=false,    % show bookmarks bar?
    pdftitle={FP PROJECT REPORT},    % title
    pdfauthor={Saptarishi Dhanuka},                     % author
    pdfsubject={TeX and LaTeX},                        % subject of the document
    pdfkeywords={TeX, LaTeX, graphics, images}, % list of keywords
    colorlinks=true,       % false: boxed links; true: colored links
    linkcolor=blue,       % color of internal links
    citecolor=black,       % color of links to bibliography
    filecolor=black,        % color of file links
    urlcolor=purple,        % color of external links
    linktoc=page            % only page is linked
}%
\def\myversion{1.0 }
\date{}
%\title
\usepackage{hyperref}
\begin{document}

\begin{flushright}
    \rule{16cm}{5pt}\vskip1cm
    \begin{bfseries}
        \Huge{FUNCTIONAL PROGRAMMING \\ CS-IS-2010-1 \\ MIDTERM EVALUATION \\ PROJECT REPORT}\\
        \vspace{1.9cm}
        Haskell Web Scraper\\
        \vspace{1.9cm}
        Saptarishi Dhanuka\\
        \vspace{1.9cm}
        Ashoka University\\
    \end{bfseries}
\end{flushright}

\tableofcontents



\chapter{Problem Statement and Requirements}
\begin{tcolorbox}[colback=white,colframe=gray,title={Assigned Project Statement}]
    Develop a scraper using Haskell to extract text and code snippets separately.
    \begin{enumerate}
        \item \textbf{Input}: Scrape the text and code snippets from the given text \href{https://eli.thegreenplace.net/2018/type-erasure-and-reification/}{source}
        \item \textbf{Output}: A Word document containing the text and \texttt{.txt} file containing the code.
        \item \textbf{Method}: Write the algorithm to scrape (you can use the \texttt{tagsoup} library) and all the input-output facilities using Haskell. Do not use any other language.
    \end{enumerate}
\end{tcolorbox}

% in the design section can give the basic structure of the HTML page

\section{Problem Description}
The given web page is made of text and code snippets, which we need to scrape and extract separately into a \texttt{.docx} file containing the text portions and a \texttt{.txt} file which has the code snippets. \\ For this, we need to fetch the given web page, parse and analyze its HTML structure to identify the HTML tags of the code snippets and the tags of the rest of the text, so that we can effectively separate them into different documents.


\section{Requirements}

\begin{enumerate}
    \item The scraper shall be written entirely in Haskell
    \item The scraper shall get all text of the given page and write it into a \texttt{.docx} file.
    \item The scraper shall get all the code snippets of the given page and write it into a \texttt{.txt} file.
\end{enumerate}



% \begin{enumerate}
%     \item \textbf{Functional}
%     \begin{enumerate}
%         \item The scraper will be written entirely in Haskell
%         \item The scraper will accurately get all text of the given page and write it in a formatted manner into the \texttt{.txt} file. It will preserve the order and structure of the text on the page
%         \item The scraper will accurately get the code snippets of the given page and write it in ordered manner into a \texttt{.docx} file.
%     \end{enumerate}
    
%     \item \textbf{Non-functional}
%     \begin{enumerate}
%         \item The scraper will gracefully handle any errors that occur during it's running
%     \end{enumerate}
        
% \end{enumerate}



\chapter{Specifications}


\begin{enumerate}
    \item The scraper will use Haskell HTTP libraries for fetching the HTML content of the given web page.
    \item The scraper will separate the code snippets from the rest of the textual content using an algorithm that utilizes the \texttt{tagsoup} library to parse the HTML content, along with other standard libraries for string and text handling.
    \item The scraper will write the text into a Word document and the code snippets into a \texttt{.txt} file mainly using the \texttt{tagsoup} and \texttt{pandoc} libraries, along with some standard Haskell libraries.
    \item The \texttt{.txt} file will be formatted such that the code snippets are visibly delimited for readability purposes.
    \item The \texttt{.docx} file will be formatted in a mannger similar to the original web page in terms of demarcating headings, footnotes and the order of the text. 
    \item The scraper will handle errors gracefully.
\end{enumerate}




% \begin{enumerate}
%     \item \textbf{Limitations} % should limitations go here or later on cause it's seeming out of place here
%     % this can go in the design and architecture part or in the specs part
%     \begin{enumerate}
%         \item Since the HTML structure of different web pages can vary, it is not necessary that this particular scraper will work for all web pages. It is designed specifically for the given page and may work for some other pages. But no generalisation can be made about the correctness of its text and code extraction for other pages.
%         \item Moreover, it will not necessarily work for web pages with malformed HTML or different structure.
%         \item It is not designed to be robust to design changes, which is in line with the \href{https://hackage.haskell.org/package/tagsoup-0.6/src/tagsoup.htm#:~:text=Rule%202%3A%0ADo%20not%20be%20robust%20to%20design%20changes%2C%20do%20not%20even%20consider%20the%20possibility%20when%20writing%20the%20code.}{rule} stated on the \texttt{tagsoup} library's documentation example. If the site's HTML structure changes, for instance if the code snippets change from being enclosed in \texttt{<pre>} tags to \texttt{<code>} tags, then the scraper will not be able to separate out the code from the text and extract them accurately.
%     \end{enumerate}
% \end{enumerate}






\chapter{Design and Architecture}
% need to put functions into Lib and main stuff in main
% mention directory structure here along with basic ideas of the functions and code what it's doing


\section{High-Level Architecture}

% insert image here
\begin{figure}[h]
    \centering
    \includegraphics[width=1.0\textwidth]{figures/high-arch.png}
    \caption{High-Level Architecture}
    \label{fig:high-level-arch}
\end{figure}

The \textbf{high-level architecture} consists of the following:
\begin{enumerate}
    \item Functionality for getting the target web page using HTTP libraries.
    \item Parsing the response obtained from the HTTP libraries into Tags from the \texttt{tagsoup} library.
    \item Separating the text from the code snippets using the descriptions of each Tag from the above Tags
    \item Extracting the visible content from the text tags and writing them into a \texttt{.docx} file
    \item Extracting the visible content from the code tags and writing them into a \texttt{.txt} file
\end{enumerate}



\section{High-Level Design}


\begin{figure}[h]
    \centering
    \includegraphics[width=1.0\textwidth]{figures/high-design.png}
    \caption{High-Level Design with functions in highlighted in cyan}
    \label{fig:high-level-design}
\end{figure}

The \textbf{high-level design} which implements the above architecture consists of the following:
\begin{enumerate}
    \item A TLS manager for handling HTTPS requests, since the given URL is prefixed with \texttt{https}
    \item Parsing the url into a request
    \item Executing the request with the TLS manager
    \item Get the body i.e. the HTML content from the response received after executing the request
    \item Parse the HTML into a list of Tag Strings according to the \texttt{tagsoup} library
    \item Separate the Tags corresponding to the code from the Tags corresponding to the textual content. By inspecting the HTML, we can see that the code snippets are within \texttt{<pre>} tags, so we need to separate everything enclosed within these tags from the rest of the HTML content
    \item Insert delimiters between each \texttt{<pre>} tag for formatting purposes. 
    \item Convert the list of Tag Strings corresponding to the code and to the text each back into an HTML-formatted string, which we then convert into a \texttt{pandoc} document as intermediate representation
    \item Convert the pandocs into another intermediate string-like format which can then be written into the respective \texttt{.docx} and \texttt{.txt} files
\end{enumerate}


% \section{Low-level design}?????????????????????











\chapter {Tools and Languages}
% tech stack
% need to be clear on cabal stack and haskell and stuff


\section{Languages}

Only Haskell was used for the project as mentioned in the problem statement

\section{Tools}

\begin{enumerate}
    \item The Glasgow Haskell Compiler (GHC) is used for compilation.
    \item \href{https://docs.haskellstack.org/en/stable/}{Stack} is used as the build tool. This manages installing project dependencies, building and running the project and testing the project.
    \item Libraries:
    \begin{enumerate}
        \item \texttt{Network.HTTP.Client.TLS} was chosen for handling HTTPS connections in order to use the \texttt{newTlsManager} function
        \item \texttt{Network.HTTP.Client} was chosen for parsing the url into a request, executing the request with the manager, and getting the body from the response. The functions used were \texttt{parseRequest, httpLbs, responseBody} respectively.
        \item \texttt{Tagsoup} was used to parse the HTML into a list of Tag Strings, and then also to convert the separated Tag Strings back into an HTML-formatted string . It was also used in a helper function that inserted a delimiter between the code snippets. The functions used were \texttt{parseTags, isTagOpenName, isTagCloseName, renderTags}, along with the \texttt{TagText} constructor and \texttt{Tag} for \texttt{Tag String}.
        \item \texttt{Pandoc} was used to 
        






        \item \texttt{Data.ByteString.Lazy.Char8} was used to convert the ByteString obtained from the response body into a string for further processing. The function used was \texttt{unpack}
        \item 
    \end{enumerate} 
\end{enumerate}






\chapter{Test Plan}
% test individual functions 




\chapter{Prototype Implementation Details}
% what the prototype is outputting and stuff without formatting and stuff




\chapter{Plan for Completion}



















\end{document}