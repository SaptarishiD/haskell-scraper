\documentclass{scrreprt}
\usepackage{listings}
\usepackage{underscore}
\usepackage[bookmarks=true]{hyperref}
\usepackage[utf8]{inputenc}
\usepackage[english]{babel}

\usepackage{tcolorbox}
\usepackage{hyperref}

\hypersetup{
    bookmarks=false,    % show bookmarks bar?
    pdftitle={FP PROJECT REPORT},    % title
    pdfauthor={Saptarishi Dhanuka},                     % author
    pdfsubject={TeX and LaTeX},                        % subject of the document
    pdfkeywords={TeX, LaTeX, graphics, images}, % list of keywords
    colorlinks=true,       % false: boxed links; true: colored links
    linkcolor=blue,       % color of internal links
    citecolor=black,       % color of links to bibliography
    filecolor=black,        % color of file links
    urlcolor=purple,        % color of external links
    linktoc=page            % only page is linked
}%
\def\myversion{1.0 }
\date{}
%\title
\usepackage{hyperref}
\begin{document}

\begin{flushright}
    \rule{16cm}{5pt}\vskip1cm
    \begin{bfseries}
        \Huge{FUNCTIONAL PROGRAMMING \\ CS-IS-2010-1 \\ MIDTERM EVALUATION \\ PROJECT REPORT}\\
        \vspace{1.9cm}
        Haskell Web Scraper\\
        \vspace{1.9cm}
        Saptarishi Dhanuka\\
        \vspace{1.9cm}
        Ashoka University\\
    \end{bfseries}
\end{flushright}

\tableofcontents



\chapter{Problem Statement}
\begin{tcolorbox}[colback=white,colframe=gray,title={Assigned Project Statement}]
    Develop a scraper using Haskell to extract text and code snippets separately.
    \begin{enumerate}
        \item \textbf{Input}: Scrape the text and code snippets from the given text \href{https://eli.thegreenplace.net/2018/type-erasure-and-reification/}{source}
        \item \textbf{Output}: A Word document containing the text and \texttt{.txt} file containing the code.
        \item \textbf{Method}: Write the algorithm to scrape (you can use the \texttt{tagsoup} library) and all the input-output facilities using Haskell. Do not use any other language.
    \end{enumerate}
\end{tcolorbox}

% in the design section can give the basic structure of the HTML page

\section{Problem Description}
The given web page is made of text and code snippets, which we need to scrape and extract separately into a \texttt{.docx} file containing the text portions and a \texttt{.txt} file which has the code snippets. \\ For this, we need to fetch the given web page, parse and analyze its HTML structure to identify the HTML tags of the code snippets and the tags of the rest of the text, so that we can effectively separate them into different documents.




\chapter{Requirements and Specifications}

\section{Requirements}

\begin{enumerate}
    \item \textbf{Functional}
    \begin{enumerate}
        \item The scraper will be written entirely in Haskell
        \item The scraper will accurately get all text of the given page and write it in a formatted manner into the \texttt{.txt} file. It will preserve the order and structure of the text on the page
        \item The scraper will accurately get the code snippets of the given page and write it in ordered manner into a \texttt{.docx} file.
    \end{enumerate}
    
    \item \textbf{Non-functional}
    \begin{enumerate}
        \item The scraper will gracefully handle any errors that occur during it's running
    \end{enumerate}
        
\end{enumerate}


\section{Specifications}

\begin{enumerate}
    \item The scraper will mainly utilise the \texttt{tagsoup} and \texttt{scalpel} libraries for the parsing and extraction of the HTML content and separation of the text and code snippets
    \item The scraper will use the HTTP libraries for fetching the given web page
    \item The scraper will use the standard module \texttt{Prelude} for writing the code snippets data into a \texttt{.txt} file
    \item The scraper will use \texttt{Pandoc} library for writing the textual data into a \texttt{.docx} file
    \item \textbf{Limitations} % should limitations go here or later on cause it's seeming out of place here
    \begin{enumerate}
        \item Since the HTML structure of different web pages can vary, it is not necessary that this particular scraper will work for all web pages. It is designed specifically for the given page and may work for some other pages. But no generalisation can be made about the correctness of its text and code extraction for other pages.
        \item Moreover, it will not necessarily work for web pages with malformed HTML or different structure.
        \item It is not designed to be robust to design changes, which is in line with the \href{https://hackage.haskell.org/package/tagsoup-0.6/src/tagsoup.htm#:~:text=Rule%202%3A%0ADo%20not%20be%20robust%20to%20design%20changes%2C%20do%20not%20even%20consider%20the%20possibility%20when%20writing%20the%20code.}{rule} stated on the \texttt{tagsoup} library's documentation example. If the site's HTML structure changes, for instance if the code snippets change from being enclosed in \texttt{<pre>} tags to \texttt{<code>} tags, then the scraper will not be able to separate out the code from the text and extract them accurately.
    \end{enumerate}
\end{enumerate}






\chapter{Design and Architecture}
% need to put functions into Lib and main stuff in main
% mention directory structure here along with basic ideas of the functions and code what it's doing




\chapter {Choice of Tools, Platforms, and Languages}
% tech stack
% need to be clear on cabal stack and haskell and stuff




\chapter{Test Plan}
% test individual functions 




\chapter{Prototype Implementation Details}
% what the prototype is outputting and stuff without formatting and stuff




\chapter{Plan for Completion}



















\end{document}